%%%%%%%%%%%%%%%%%%%%%%%%%%%%%%%%%%%%%%%%%
% Large Colored Title Article
% LaTeX Template
% Version 1.1 (25/11/12)
%
% This template has been downloaded from:
% http://www.LaTeXTemplates.com
%
% Original author:
% Frits Wenneker (http://www.howtotex.com)
%
% License:
% CC BY-NC-SA 3.0 (http://creativecommons.org/licenses/by-nc-sa/3.0/)
%
%%%%%%%%%%%%%%%%%%%%%%%%%%%%%%%%%%%%%%%%%

%----------------------------------------------------------------------------------------
%	PACKAGES AND OTHER DOCUMENT CONFIGURATIONS
%----------------------------------------------------------------------------------------

\documentclass[DIV=calc, paper=a4, fontsize=11pt, twocolumn]{scrartcl}	 % A4 paper and 11pt font size

\usepackage{lipsum} % Used for inserting dummy 'Lorem ipsum' text into the template
\usepackage[german]{babel} % English language/hyphenation
\usepackage[protrusion=true,expansion=true]{microtype} % Better typography
\usepackage{amsmath,amsfonts,amsthm} % Math packages
\usepackage[svgnames]{xcolor} % Enabling colors by their 'svgnames'
\usepackage[hang, small,labelfont=bf,up,textfont=it,up]{caption} % Custom captions under/above floats in tables or figures
\usepackage{booktabs} % Horizontal rules in tables
\usepackage{fix-cm}	 % Custom font sizes - used for the initial letter in the document

\usepackage{sectsty} % Enables custom section titles
\allsectionsfont{\usefont{OT1}{phv}{b}{n}} % Change the font of all section commands

\usepackage{fancyhdr} % Needed to define custom headers/footers
\pagestyle{fancy} % Enables the custom headers/footers
\usepackage{lastpage} % Used to determine the number of pages in the document (for "Page X of Total")
% for urls
\usepackage{url}
\usepackage{hyperref}

% Headers - all currently empty
\lhead{}
\chead{}
\rhead{}

% Footers
\lfoot{}
\cfoot{}
\rfoot{\footnotesize Page \thepage\ of \pageref{LastPage}} % "Page 1 of 2"

\renewcommand{\headrulewidth}{0.0pt} % No header rule
\renewcommand{\footrulewidth}{0.4pt} % Thin footer rule

\usepackage{lettrine} % Package to accentuate the first letter of the text
\newcommand{\initial}[1]{ % Defines the command and style for the first letter
\lettrine[lines=3,lhang=0.3,nindent=0em]{
\color{DarkBlue}
{\textsf{#1}}}{}}

%----------------------------------------------------------------------------------------
%	TITLE SECTION
%----------------------------------------------------------------------------------------

\usepackage{titling} % Allows custom title configuration

\newcommand{\HorRule}{\color{DarkBlue} \rule{\linewidth}{1pt}} % Defines the gold horizontal rule around the title

\pretitle{\vspace{-30pt} \begin{flushleft} \HorRule \fontsize{50}{50} \usefont{OT1}{phv}{b}{n} \color{DarkBlue} \selectfont} % Horizontal rule before the title

\title{Wortdefinitionen WARB} % Your article title

\posttitle{\par\end{flushleft}\vskip 0.5em} % Whitespace under the title

\preauthor{\begin{flushleft}\large \lineskip 0.5em \usefont{OT1}{phv}{b}{sl} \color{DarkBlue}} % Author font configuration

\author{longstone, } % Your name

\postauthor{\footnotesize \usefont{OT1}{phv}{m}{sl} \color{Black} % Configuration for the institution name
ZHAW School of Management and Law  % Your institution

\par\end{flushleft}\HorRule} % Horizontal rule after the title

\date{} % Add a date here if you would like one to appear underneath the title block

%----------------------------------------------------------------------------------------

\begin{document}

\maketitle % Print the title

\thispagestyle{fancy} % Enabling the custom headers/footers for the first page 

%----------------------------------------------------------------------------------------
%	ABSTRACT
%----------------------------------------------------------------------------------------

\subsection*{Qualitative Methode}
Bsp: Interviews. Empirisch, kleine St\"uckzahl

\subsection*{Quantitative Methode}
Bsp: Umfragen. Statistische Methode, grosse St\"uckzahl

\subsection*{Design Science}
Methode der Wirtschaftsinformatik

\subsection*{Proposition}
'Neuer Begriff': wage allgemeine Hypothese (Vorstufe zur Hypothese)

\subsection*{Hypothese}
Vorstellung, mittels wissenschaftlichem zu Projekt \"uberpr\"ufen

\subsection*{Habermas}
Philosophie geh\"ort zum kritischen Rationalismus.

\subsection*{Karl Popper}
Theorie des empirischen Falsifikationsprinzips

\subsection*{Simulation}
Methode um Prototypen zu validieren (eine Simulation kann ein Prototyp sein)

\subsection*{Expertenbefragung}
in der empirische Forschung, zur Validieren

\subsection*{Relevanz}
bedeutet das die Forschung in der Praxis n\"utzlich/von belang ist

\subsection*{Paradigma}
die 2 Paradigmen der Wirtschaftsinformatik sind Gestalt- und Designorientiert

\subsection*{Rigorosit\"at}
Es geht in erster Linie um die Nachvollziehbarkeit der (gef\"uhrten) Forschung. Um Nutzung von anerkannten, validen Methoden, die zur Forschungsfrage passen.
\subsection*{Meinung}
Meinung interessiert uns nicht in der Forschung.

\subsection*{Behaviorism (Verhaltensorientiert)}
Aus Sicht der WI: Paradigma das sich mit dem verhalten der menschen im Kontext zu Informationssystemen besch\"aftigt

\subsection*{Empirie}
Qualitativ, Quantitativ um verhalten zu untersuchen (in der Regel).
Empirie [ɛmpiˈʀiː][1] (von griechisch εμπειρία empeiría ‚Erfahrung, Erfahrungswissen‘) ist eine methodische Sammlung von Daten. Auch die Erkenntnisse aus empirischen Daten werden manchmal kurz Empirie genannt.\cite{Wikipedia:2016a}
\subsection*{Reputation}
Rigorosit\"at / Nachvollziehbarkeit sch\"utzt vor Reputationsverlust

\subsection*{Erkenntnisobjekt}
Gegenstand der Forschung

\subsection*{Validieren}
Forschungsergebnisse sollten Validiert werden.

\subsection*{Literaturanalyse}
Ben\"otigt f\"ur state of the art. Muss systematisch und nachvollziehbar erfolgen. Nutzung von relevanter Literatur.

\subsection*{Praxis}
siehe auch Relevanz: sinnvoll alles was erforscht wird ist n\"utzlich f\"ur die Praxis

\subsection*{Wissenschaftlichkeit}
Master thesis sollte die Kriterien der Wissenschaftlichkeit erf\"ullten

\subsection*{Pluralismus}
ich verwende verschiedene Methoden um meine Forschung m\"oglichst rigoros darzustellen

\subsection*{Gestaltorientierung (Design Science)}
Erstellt ein Artefakt nach anerkannten Methoden z.B. Hevner et al.
\subsection*{Triangulation}
mehrere Methoden (mehr als Pluralismus) verwenden um zu Validieren

\subsection*{Fallstudie}
Empirisch, verhaltensorientierte Forschung. Soll auch methodisch fundiert sein. Methode die in der Wirtschaftsinformatik verbreitet ist.

\subsection*{Wissen}
im Gegensatz zur Meinung, nachvollziehbar nicht falsifiziert

\subsection*{Methode}
Vorgehen welches zu unserer Forschung passt

\subsection*{Hermeneutik}
Methode zur Analyse von Texten.

\subsection*{Theorie}
Theorien können aus unterschiedlichen Disziplinen stammen. Die Theorie kann auch Forschungsobjekt sein, um eine neue Theorie zu entwickeln.
Grundlage um Hypothese zu pr\"ufen.

\subsection*{Interpretation}
Analyse von Texten / Interviews, immer in Abh\"angigkeit mit dem sozialen Kontext

\subsubsection*{Induktion}
vom Einzelfall auf eine allgemeing\"ultige Aussage schliessen

\subsubsection*{Deduktion}
von der Allgemeing\"ultigkeit zum Einzelfall

\begin{thebibliography}{99} % Bibliography - this is intentionally simple in this template

\bibitem[Asprion, 2015]{Asprion:2015}
Prof. Dr. Petra Asprion
\newblock Wissenschaftstheorie
\newblock {\em Modul WARB, MSc Wirtschaftsinformatik}
 
 \bibitem[Slobi, 2015]{Slobi:2015}
BSc Computer Science Slobi S
\newblock Handnotizen
\newblock {\em Modul WARB, MSc Wirtschaftsinformatik}

 \bibitem[Cuche, 2015]{Cuche:2015}
BSc Computer Science Cuche S
\newblock Handnotizen
\newblock {\em Modul WARB, MSc Wirtschaftsinformatik}
 
 \bibitem[Wikipedia, 2016a]{Wikipedia:2016a}
 Wikipedia
 \newblock Empirie --- Wikipedia{,} Die freie Enzyklopädie
\newblock {\em [Online; Stand 13. Januar 2016] \url{https://de.wikipedia.org/w/index.php?title=Empirie&oldid=147179666}}
\end{thebibliography}

%----------------------------------------------------------------------------------------

\end{document}